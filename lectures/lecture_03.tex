\part{Lecture 3}

\section{Q3.1 Define binary classification, write down the perceptron algorithm, and show how a prediction is made for a given data instance $x$. [10]}

Binary classification is the task of classifying the elements of a given set into two groups based on a classification rule. In binary classification, the output variable can take only two values, typically denoted as 0 and 1, or -1 and 1 in some contexts.

The perceptron algorithm is a binary classifier that linearly separates these two classes. The algorithm iteratively adjusts the weights based on the training data. Given a set of features \( x \) and a target \( t \), the perceptron rule updates the weights \( w \) as follows:

\begin{algorithm}[H]
\caption{Perceptron Learning Algorithm}
\begin{algorithmic}[1]
\Require Linearly separable dataset $X\in\mathbb{R}^{N\times D}$, labels $\boldsymbol t\in\{-1,+1\}^N$
\Ensure Weights $\boldsymbol w\in\mathbb{R}^D$ such that $t_i\,\boldsymbol x_i^\top \boldsymbol w>0$ for all $i$
\State $\boldsymbol w \leftarrow \boldsymbol 0$
\While{not all examples are classified correctly}
    \For{$i=1$ \textbf{to} $N$}
        \State $y \leftarrow \boldsymbol x_i^\top \boldsymbol w$
        \If{$t_i\,y \le 0$} \Comment{misclassified example}
            \State $\boldsymbol w \leftarrow \boldsymbol w + t_i\,\boldsymbol x_i$
        \EndIf
    \EndFor
\EndWhile
\end{algorithmic}
\end{algorithm}

To make a prediction \( \hat{y} \) for a new example with feature vector \( x \), the perceptron uses the sign of the dot product between the features and weights:

\[
\hat{y} = \text{sign}(x^T w)
\]

where \( \text{sign} \) is an activation function that maps positive values to +1 and non-positive values to -1.

\subsection*{Prediction Example}
Given a new input \( x \) and trained weights \( w \), the perceptron prediction is computed as:
\[
\hat{y} = \text{sign}(x^T w + b)
\]
where \( b \) is the bias term of the perceptron. If \( \hat{y} \) is positive, the input is classified into one class, and if it is negative, it is classified into the other class.

\section{Q3.2 Explain what it means for a dataset to be linearly separable. Give an example of a simple 2D dataset that is \textit{not} linearly separable and explain why the perceptron algorithm would fail on it. [10]}
A binary-labeled dataset $\{(\boldsymbol x_i,t_i)\}_{i=1}^N$ with $t_i\in\{-1,+1\}$ is called \textbf{linearly separable}
if there exist parameters $(\boldsymbol w,b)$ such that a single linear decision boundary separates the two classes:
\[
\exists\,(\boldsymbol w,b)\ \ \text{s.t.}\ \ t_i\bigl(\boldsymbol w^\top \boldsymbol x_i + b\bigr) > 0 \quad \forall i.
\]
Equivalently, all positive examples lie in one open half-space and all negative examples lie in the other half-space
induced by the hyperplane (line in 2D) $\boldsymbol w^\top \boldsymbol x + b = 0$.

\subsection*{Example of a simple 2D dataset that is \emph{not} linearly separable}
A standard example is the XOR pattern with four points in $\mathbb{R}^2$:
\[
\text{Positive }(+1):\ (0,0),\ (1,1),
\qquad
\text{Negative }(-1):\ (0,1),\ (1,0).
\]
This set is \emph{not} linearly separable because any line divides the plane into two half-planes, but in XOR each class
occupies two opposite corners of a square; whichever side of a line contains $(0,0)$ will necessarily contain at least one
negative point (or vice versa). Therefore, no single linear boundary can make all signs correct simultaneously.

\subsection*{Why the perceptron fails on such data}
The perceptron algorithm iteratively updates $\boldsymbol w$ whenever it finds a misclassified example, and its convergence
guarantee relies on the existence of a separating hyperplane (linear separability). If the dataset is not linearly separable,
there is \emph{no} $\boldsymbol w$ (and $b$) satisfying $t_i(\boldsymbol w^\top \boldsymbol x_i + b)>0$ for all $i$, so the
algorithm keeps encountering misclassified points and continues updating indefinitely (it does not terminate/converge).

\section{Q3.3 For discrete random variables, define entropy, cross-entropy, and Kullback-Leibler divergence, and prove the Gibbs inequality (i.e., that KL divergence is non-negative). [20]}

Entropy \( H(P) \) for a discrete random variable with probability distribution \( P \) is defined as:
\[
H(P) = - \sum_x P(x) \log P(x)
\]
It measures the expected level of 'surprise' or uncertainty inherent in the variable's possible outcomes.

Cross-entropy \( H(P, Q) \) between two discrete probability distributions \( P \) and \( Q \) is defined as:
\[
H(P, Q) = - \sum_x P(x) \log Q(x)
\]
It measures the expected number of bits (if the log is in base 2) required to identify an event from a set of possibilities if a wrong distribution \( Q \) is used instead of the true distribution \( P \).

Kullback-Leibler divergence \( D_{KL}(P||Q) \) from \( Q \) to \( P \) is defined as:
\[
D_{KL}(P||Q) = H(P, Q) - H(P) = \sum_x P(x) \log \frac{P(x)}{Q(x)}
\]
It measures how one probability distribution diverges from a second, expected probability distribution.

\textbf{Proof of Gibbs Inequality:}
We want to prove that \( D_{KL}(P||Q) \geq 0 \), with equality if and only if \( P = Q \).

Using the log sum inequality \( \log \frac{a}{b} \leq \frac{a}{b} - 1 \) with equality only if \( a = b \), we have:

\begin{align*}
H(P) - H(P,Q) &= \sum_x P(x) \log \frac{Q(x)}{P(x)} \\
&\leq \sum_x P(x) \left( \frac{Q(x)}{P(x)} - 1 \right) \\
&= \sum_x Q(x) - \sum_x P(x) \\
&= 0
\end{align*}

since \( \sum_x P(x) = 1 \) and \( \sum_x Q(x) = 1 \). The inequality is strict unless \( P(x) = Q(x) \) for all \( x \), which proves Gibbs Inequality.


\section{Q3.4 Explain the notion of likelihood in machine learning. What likelihood are we estimating, and why do we do it? [10]}

Likelihood in the context of maximum likelihood estimation (MLE) is a function that measures the probability of observing the given data under different parameter values of a statistical model. For a set of independent and identically distributed (i.i.d.) data points \( X = \{x_1, x_2, \ldots, x_N\} \), the likelihood of a parameter \( w \) is defined as:

\[
L(w) = \prod_{i=1}^{N} P_{\text{model}}(x_i; w)
\]

where \( P_{\text{model}}(x_i; w) \) is the probability of observing the specific data point \( x_i \) under the model parameterized by \( w \).

In MLE, we seek the parameter \( w \) that maximizes this likelihood function, which is equivalent to maximizing the probability of observing the given data. While the likelihood itself is not a probability distribution, it serves as a scoring function that indicates how well the model with a particular set of parameters explains the observed data. Maximizing the likelihood function leads to finding the parameter values that make the observed data most probable under the assumed model.


\section{Q3.5 Describe maximum likelihood estimation as minimizing NLL, cross-entropy, and KL divergence and explain whether they differ or are the same and why. [20]}
Let $x_1,\dots,x_N$ be i.i.d.\ samples from an (unknown) data distribution $p_{\text{data}}(x)$.
We fit a probabilistic model $p_{\text{model}}(x;w)$.

\subsection*{1) Maximum likelihood $\Leftrightarrow$ minimizing negative log-likelihood (NLL)}
Maximum likelihood estimation chooses parameters that maximize the likelihood of the observed dataset:
\[
w_{\text{MLE}}
= \arg\max_w \prod_{i=1}^{N} p_{\text{model}}(x_i;w).
\]
Taking the logarithm (monotone) and negating turns this into minimization of the \textbf{negative log-likelihood}:
\[
w_{\text{MLE}}
= \arg\min_w \left[-\sum_{i=1}^{N} \log p_{\text{model}}(x_i;w)\right].
\]
Often we use the average NLL (same minimizer, just scaled by $1/N$):
\[
\mathcal{L}_{\text{NLL}}(w)= -\frac{1}{N}\sum_{i=1}^{N} \log p_{\text{model}}(x_i;w).
\]

\subsection*{2) NLL $\Leftrightarrow$ cross-entropy with the empirical distribution}
Define the empirical distribution $\hat p(x)$ that assigns probability $1/N$ to each observed sample (and $0$ elsewhere).
Then the average NLL can be written as an expectation:
\[
\mathcal{L}_{\text{NLL}}(w)
= \mathbb{E}_{x\sim \hat p}\bigl[-\log p_{\text{model}}(x;w)\bigr].
\]
By definition, the \textbf{cross-entropy} between distributions $p$ and $q$ is
\[
H(p,q)=\mathbb{E}_{x\sim p}\bigl[-\log q(x)\bigr].
\]
Therefore,
\[
\boxed{\ \mathcal{L}_{\text{NLL}}(w) = H(\hat p,\;p_{\text{model}}(x;w))\ }.
\]
So, for a fixed dataset, \textbf{minimizing average NLL is exactly the same objective as minimizing cross-entropy with $\hat p$.}

\subsection*{3) Cross-entropy $\Leftrightarrow$ KL divergence (up to an additive constant)}
The KL divergence is
\[
D_{\mathrm{KL}}(p\|q)=\mathbb{E}_{x\sim p}\left[\log\frac{p(x)}{q(x)}\right]
= \mathbb{E}_{x\sim p}[\log p(x)]-\mathbb{E}_{x\sim p}[\log q(x)].
\]
Using $H(p)=\mathbb{E}_{x\sim p}[-\log p(x)]$ and $H(p,q)=\mathbb{E}_{x\sim p}[-\log q(x)]$, we get the identity
\[
\boxed{\ H(p,q) = H(p) + D_{\mathrm{KL}}(p\|q)\ }.
\]
Applying this with $p=\hat p$ and $q=p_{\text{model}}(x;w)$ yields
\[
H(\hat p,\;p_{\text{model}}(x;w))
= H(\hat p) + D_{\mathrm{KL}}\!\left(\hat p \,\|\, p_{\text{model}}(x;w)\right).
\]

\subsection*{4) Are NLL, cross-entropy, and KL the same or different?}
\begin{itemize}
    \item \textbf{NLL vs.\ cross-entropy:} for a fixed dataset, \emph{they are the same objective} (NLL is cross-entropy with the empirical distribution $\hat p$), possibly differing only by a constant scale factor ($1$ vs.\ $1/N$), which does not change the minimizer.
    \item \textbf{Cross-entropy vs.\ KL:} they differ by an \emph{additive constant} $H(\hat p)$ that does \emph{not} depend on $w$.
    Hence,
    \[
    \arg\min_w H(\hat p,p_{\text{model}}(\cdot;w))
    \;=\;
    \arg\min_w D_{\mathrm{KL}}(\hat p \,\|\, p_{\text{model}}(\cdot;w)).
    \]
\end{itemize}
Therefore, \textbf{MLE can be viewed equivalently as minimizing NLL, minimizing cross-entropy, or minimizing KL divergence},
because these objectives are identical up to scaling and/or adding constants independent of $w$.

\subsection*{5) Conditional (supervised) case}
For supervised learning with pairs $(x_i,t_i)$ and model $p_{\text{model}}(t\mid x;w)$:
\[
w_{\text{MLE}}
= \arg\min_w \left[-\sum_{i=1}^{N}\log p_{\text{model}}(t_i\mid x_i;w)\right]
= \arg\min_w \ \mathbb{E}_{(x,t)\sim \hat p}\bigl[-\log p_{\text{model}}(t\mid x;w)\bigr].
\]
This is the \textbf{conditional cross-entropy} between the empirical conditional distribution and the model, and it equals
a conditional KL divergence plus an additive constant independent of $w$.

{\color{gray}

Let \( X = \{x_1, x_2, \ldots, x_N\} \) be training data drawn independently from the data-generating distribution \( p_{\text{data}} \). We denote the empirical data distribution as \( \hat{p}_{\text{data}} \), where \( \hat{p}_{\text{data}}(x) = \frac{1}{N} \sum_{i=1}^{N} \mathbb{I}[x_i = x] \). Let \( p_{\text{model}}(x; w) \) be a family of distributions.

The maximum likelihood estimation of \( w \) is:

\[
w_{\text{MLE}} = \arg \max_w p_{\text{model}}(X; w) = \arg \max_w \prod_{i=1}^{N} p_{\text{model}}(x_i; w)
\]

\[
= \arg \min_w -\sum_{i=1}^{N} \log p_{\text{model}}(x_i; w)
\]

\[
= \arg \min_w \mathbb{E}_{x \sim \hat{p}_{\text{data}}}[- \log p_{\text{model}}(x; w)]
\]

\[
= \arg \min_w H(\hat{p}_{\text{data}}(x), p_{\text{model}}(x; w))
\]

\[
= \arg \min_w D_{KL}(\hat{p}_{\text{data}}(x) || p_{\text{model}}(x; w)) + H(\hat{p}_{\text{data}}(x))
\]

For MLE generalized to the conditional case, where the goal is to predict \( t \) given \( x \):

\[
w_{\text{MLE}} = \arg \max_w p_{\text{model}}(t | x; w) = \arg \max_w \prod_{i=1}^{N} p_{\text{model}}(t_i | x_i; w)
\]

\[
= \arg \min_w -\sum_{i=1}^{N} \log p_{\text{model}}(t_i | x_i; w)
\]

\[
= \arg \min_w \mathbb{E}_{(x,t) \sim \hat{p}_{\text{data}}}[- \log p_{\text{model}}(t | x; w)]
\]

\[
= \arg \min_w H(\hat{p}_{\text{data}}(t | x), p_{\text{model}}(t | x; w))
\]

\[
= \arg \min_w D_{KL}(\hat{p}_{\text{data}}(t | x) || p_{\text{model}}(t | x; w)) + H(\hat{p}_{\text{data}}(t | x))
\]

Where \( H(\hat{p}_{\text{data}}) \) is the entropy of the empirical data distribution and \( D_{KL} \) is the Kullback-Leibler divergence. The terms are defined such that the conditional entropy is \( H(\hat{p}_{\text{data}}) = \mathbb{E}_{(x,t) \sim \hat{p}_{\text{data}}}[-\log(\hat{p}_{\text{data}}(t | x))] \) and the conditional cross-entropy is \( H(\hat{p}_{\text{data}}, p_{\text{model}}) = \mathbb{E}_{(x,t) \sim \hat{p}_{\text{data}}}[-\log(p_{\text{model}}(t | x; w))] \). The negative log-likelihood (NLL) is equivalent to cross-entropy or Kullback-Leibler divergence in the context of MLE.
}
\section{Q3.6 Provide an intuitive justification for why cross-entropy is a good optimization objective in machine learning. What distributions do we compare in cross-entropy? Why is it good when the cross-entropy is low? [5]}

Cross-entropy is a good optimization objective in machine learning because it measures how well the predicted probability distribution from a model matches the true distribution of the target labels. Intuitively, it quantifies the difference between what the model believes (its predicted probabilities) and the actual outcomes.

In classification tasks, cross-entropy compares:
\begin{itemize}
    \item True distribution: Represented as a one-hot encoded vector for each class, where the correct class has a probability of 1, and all others are 0.
    \item Predicted distribution: The model's output probabilities for each class (softmax or sigmoid outputs).
\end{itemize}

When the cross-entropy is low, it means the predicted probabilities are close to the true labels — the model is confident and correct. For instance, if a model correctly predicts a probability near 1 for the correct class and near 0 for others, the cross-entropy loss is minimal. Conversely, high cross-entropy means the predictions are far from the true labels, indicating poor performance.

In essence, minimizing cross-entropy encourages the model to assign high probabilities to correct labels, leading to better classification accuracy and more confident predictions aligned with the true data distribution.

\section{Q3.7 Considering binary logistic regression model, write down its parameters (including their size) and explain how prediction is performed (including the formula for the sigmoid function). [10]}
In binary logistic regression we model the conditional class probabilities for two classes
$C_0$ and $C_1$ using a linear score followed by a sigmoid.

\subsection*{Parameters (including sizes)}
Assume an input feature vector $\boldsymbol x \in \mathbb{R}^{D}$.
The model parameters are:
\[
\boldsymbol w \in \mathbb{R}^{D} \quad \text{(weight vector)}, 
\qquad
b \in \mathbb{R} \quad \text{(bias / intercept)}.
\]
(Equivalently, one may absorb $b$ into $\boldsymbol w$ by padding $\boldsymbol x$ with a constant $1$.)

\subsection*{Prediction (including sigmoid)}
First compute the \emph{linear part} (score / logit)
\[
\bar{y}(\boldsymbol x;\boldsymbol w,b) = \boldsymbol x^\top \boldsymbol w + b.
\]
Then apply the sigmoid function $\sigma$ to obtain a probability:
\[
y(\boldsymbol x;\boldsymbol w,b) = \sigma\!\left(\bar{y}(\boldsymbol x;\boldsymbol w,b)\right)
= \sigma(\boldsymbol x^\top \boldsymbol w + b),
\qquad
\sigma(z) = \frac{1}{1+e^{-z}}.
\]
Thus the model outputs
\[
p(C_1 \mid \boldsymbol x) = \sigma(\boldsymbol x^\top \boldsymbol w + b),
\qquad
p(C_0 \mid \boldsymbol x) = 1 - p(C_1 \mid \boldsymbol x).
\]

A hard class prediction is typically obtained by thresholding the probability, e.g.
\[
\hat{c}=
\begin{cases}
C_1, & p(C_1\mid \boldsymbol x)\ge 0.5,\\
C_0, & \text{otherwise}.
\end{cases}
\]

{\color{gray}
In a binary logistic regression model, the prediction \( \hat{y} \) is based on the probability that a given input \( x \) belongs to a particular class \( C_1 \), which is modeled using the logistic function \( \sigma \). The parameters of the model include:

\begin{itemize}
    \item Weight vector \( w \in \mathbb{R}^D \), where \( D \) is the number of features.
    \item Bias \( b \in \mathbb{R} \).
\end{itemize}

The logistic regression model makes predictions using the sigmoid function \( \sigma \) applied to the linear combination of the input features and the model weights:

\[
\sigma(z) = \frac{1}{1 + e^{-z}} = \frac{e^z}{e^z + 1}
\]

Given an input \( x \), the linear part of the logistic regression model computes \( z \) as:

\[
z = x^T w + b
\]

The final prediction \( \hat{y} \) is given by:

\[
\hat{y}(x; w) = \sigma(z) = \sigma(x^T w + b)
\]

The output of the linear part \( x^T w + b \) can be interpreted as logits, which are the log odds of the probability that \( x \) belongs to class \( C_1 \) before the sigmoid transformation. Logits can take any real value, and transforming them through the sigmoid function maps them to the \((0, 1)\) interval, representing probabilities.
}

\section{Q3.8 Write down an $L^2$-regularized minibatch SGD algorithm for training a binary logistic regression model, including the explicit formulas (i.e., formulas you would need to code it in \texttt{numpy}) of the loss function and its gradient. [20]}
To train the logistic regression, we use MLE (maximum likelihood estimation). The loss for a minibatch $\mathcal{X} = \{(\vec{x}_1, t_1), (\vec{x}_2, t_2), \ldots, (\vec{x}_N, t_N)\}$ is given by
$$
E({w}) = \frac{1}{N} \sum_{i} -\log(p(C_{t_i} | {x}_i; {w})) + \frac{\lambda}{2} \| {w} \|^2
$$

\subsection*{Model and sigmoid}
\[
p(C_1\mid \boldsymbol x;\boldsymbol w)=\sigma(\boldsymbol x^\top\boldsymbol w),
\qquad
\sigma(z)=\frac{1}{1+e^{-z}}.
\]
\subsection*{Minibatch SGD algorithm}
\begin{algorithm}[H]
\caption{$L^2$-regularized minibatch SGD for binary logistic regression}
\begin{algorithmic}[1]
\Require Input dataset $(X\in\mathbb{R}^{N\times D},\ \boldsymbol t\in\{0,1\}^N)$, learning rate $\alpha>0$, regularization $\lambda\ge 0$
\Ensure Parameters $\boldsymbol w\in\mathbb{R}^D$
\State $\boldsymbol w \leftarrow \boldsymbol 0$ \textbf{or} initialize $\boldsymbol w$ randomly
\While{until convergence (or patience runs out)}
    \State Choose a minibatch of indices $B$
    \State $\boldsymbol g \leftarrow \frac{1}{|\mathcal{B}|} \sum_{i \in \mathcal{B}} \nabla_{w} \left(-\log(p(C_{t_i} | {x}_i; {w}))\right) + \lambda {w}$
    \Comment{$\boldsymbol g = \frac{1}{|B|}X_B^\top(\sigma(X_B\boldsymbol w)-\boldsymbol t_B)+\lambda\boldsymbol w$}
    \State $\boldsymbol w \leftarrow \boldsymbol w - \alpha\,\boldsymbol g$
\EndWhile
\end{algorithmic}
\end{algorithm}

{\color{gray}

The parameters are updated using the SGD algorithm as follows:
\begin{enumerate}
  \item Initialize the weight vector $\vec{w} \leftarrow \vec{0}$ or randomly.
  \item Repeat until convergence:
  \begin{itemize}
    \item Compute the gradient for a minibatch $\mathcal{B}$:
    $$
    \vec{g} \leftarrow \frac{1}{|\mathcal{B}|} \sum_{i \in \mathcal{B}} \nabla_{\vec{w}} \left(-\log(p(C_{t_i} | \vec{x}_i; \vec{w}))\right) + \lambda \vec{w}
    $$
    \item Update the weights:
    $$
    \vec{w} \leftarrow \vec{w} - \alpha \vec{g}
    $$
  \end{itemize}
\end{enumerate}

\subsection*{Logistic Regression Gradient}
Consider the log-likelihood of logistic regression $\log p(t|\vec{x}; \vec{w})$. For brevity, we denote $\bar{y}(\vec{x}; \vec{w}) = \vec{x}^T \vec{w}$ simply as $\bar{y}$ in the following computation.

Given that for $t \sim \text{Ber}(\phi)$ we have $p(t) = \phi^t (1-\phi)^{1-t}$, we can rewrite the log-likelihood as:

\[
\log p(t|\vec{x}; \vec{w}) = \log \sigma(\bar{y})^t (1 - \sigma(\bar{y}))^{1-t}
\]

This simplifies to:

\[
t \cdot \log (\sigma(\bar{y})) + (1 - t) \cdot \log (1 - \sigma(\bar{y}))
\]

The gradient of the logistic regression's likelihood with respect to the weights $\vec{w}$ is derived as follows:
\begin{align*}
\nabla_{\vec{w}} -\log p(t|\vec{x}; \vec{w}) &= \nabla_{\vec{w}} \left( -t \log (\sigma(\hat{y})) - (1-t) \log (1 - \sigma(\hat{y})) \right) \\
&= \nabla_{\vec{w}} \left( -t \log (\sigma(\vec{x}^T \vec{w})) - (1-t) \log (1 - \sigma(\vec{x}^T \vec{w})) \right) \\
&= -t \cdot \frac{1}{\sigma(\hat{y})} \cdot \nabla_{\vec{w}} \sigma(\hat{y}) + (1-t) \cdot \frac{1}{1 - \sigma(\hat{y})} \cdot \nabla_{\vec{w}} (-\sigma(\hat{y})) \\
&= \left( -t + t \sigma(\hat{y}) + \sigma(\hat{y}) - t \sigma(\hat{y}) \right) \vec{x} \\
&= \left( \sigma(\vec{x}^T \vec{w}) - t \right) \vec{x}
\end{align*}
where $\hat{y} = \vec{x}^T \vec{w}$ and the gradient of the sigmoid function $\sigma(x)$ is $\sigma(x)(1 - \sigma(x))$. The resulting gradient is used to update the weights in the SGD algorithm.

Therefor the gradient of the loss function is:
\[
\nabla_{\vec{w}} E(\vec{w}) = \frac{1}{N} \sum_{i} \left( \sigma(\vec{x}_i^T \vec{w}) - t_i \right) \vec{x}_i + \lambda \vec{w}
\]
}

\section{Q3.9 Compare and contrast perceptron and logistic regression by discussing: (a) what each algorithm optimizes, (b) whether each provides probability estimates, (c) whether each is guaranteed to converge, and (d) the quality of solutions each finds. [10]}
Consider binary classification with inputs $\boldsymbol x\in\mathbb{R}^D$ and labels $t\in\{-1,+1\}$ (perceptron)
or $t\in\{0,1\}$ (logistic regression).

\subsection*{(a) What each algorithm optimizes}
\textbf{Perceptron:} The perceptron update is mistake-driven: when an example is misclassified
($t_i\,\boldsymbol w^\top \boldsymbol x_i \le 0$), it updates $\boldsymbol w\leftarrow \boldsymbol w+t_i\boldsymbol x_i$.
It can be viewed as minimizing the \emph{perceptron loss}
\[
\ell_{\text{perc}}(\boldsymbol w; \boldsymbol x_i,t_i)=\max\{0,-t_i\,\boldsymbol w^\top \boldsymbol x_i\},
\]
using an online/stochastic update (it does not directly maximize a likelihood). 

\textbf{Logistic regression:} Logistic regression explicitly optimizes a smooth convex objective: it minimizes the
negative log-likelihood (cross-entropy) of a Bernoulli model (often with optional $L^2$ regularization), e.g.
\[
\min_{\boldsymbol w,b}\; -\sum_{i=1}^N \Bigl[t_i\log \sigma(\boldsymbol w^\top \boldsymbol x_i+b)
+(1-t_i)\log\bigl(1-\sigma(\boldsymbol w^\top \boldsymbol x_i+b)\bigr)\Bigr] + \frac{\lambda}{2}\|\boldsymbol w\|^2.
\]

\subsection*{(b) Probability estimates}
\textbf{Perceptron:} Produces a hard decision $\operatorname{sign}(\boldsymbol w^\top \boldsymbol x+b)$; it does \emph{not}
define calibrated probabilities.

\textbf{Logistic regression:} Produces probability estimates
\[
p(C_1\mid \boldsymbol x)=\sigma(\boldsymbol w^\top \boldsymbol x+b),
\qquad \sigma(z)=\frac{1}{1+e^{-z}},
\]
so it naturally outputs class probabilities.

\subsection*{(c) Convergence guarantee}
\textbf{Perceptron:} Guaranteed to converge in a finite number of updates \emph{only if} the data are linearly separable
(perceptron convergence theorem). If the data are not separable, it may cycle and never converge.

\textbf{Logistic regression:} The (unregularized) logistic loss is convex, so gradient-based optimization converges to a
global minimum in the optimization sense (up to numerical tolerance) under standard conditions (e.g.\ appropriate step sizes).
However, if the data are perfectly separable, the MLE has no finite optimum (weights can grow without bound); $L^2$
regularization fixes this by making the objective strongly convex and ensuring a unique finite solution.

\subsection*{(d) Quality of solutions}
\textbf{Perceptron:} When separable, it finds \emph{some} separating hyperplane, but not necessarily the best one by margin
or likelihood; the final solution depends on the order of examples and updates, and can have poor generalization.

\textbf{Logistic regression:} Finds the \emph{global optimum} of the (regularized) log-likelihood / cross-entropy objective.
This typically yields more stable solutions, better-calibrated outputs, and often better generalization than perceptron,
especially on noisy or non-separable data.c

\section{Q3.10 Explain in intuitive terms why we use the logarithm when working with likelihoods in machine learning. What are the computational and optimization advantages of using negative log-likelihood instead of directly maximizing the likelihood? [5]}
Given i.i.d.\ data, the likelihood is a product of per-example probabilities,
\[
p(X;w)=\prod_{i=1}^N p(x_i;w).
\]
We use the logarithm because it turns this product into a sum,
\[
\log p(X;w)=\sum_{i=1}^N \log p(x_i;w),
\]
which has several advantages:

\begin{itemize}
    \item \textbf{Numerical stability:} Products of many probabilities (each $\le 1$) can underflow to zero in floating-point
    arithmetic, especially for large $N$. Summing log-probabilities avoids underflow and is stable.
    \item \textbf{Computational simplicity:} Sums are cheaper and easier to manipulate than products; gradients also become sums
    over examples, which naturally supports minibatch SGD.
    \item \textbf{Optimization convenience:} $\log(\cdot)$ is strictly increasing, so maximizing likelihood and maximizing
    log-likelihood have the same maximizer. Minimizing the \emph{negative} log-likelihood (NLL) is equivalent but fits the
    standard ``minimize a loss'' framework.
    \item \textbf{Better-behaved objectives:} For many common models (e.g.\ logistic regression), the NLL is smooth and convex in
    the parameters, making optimization more reliable than working with the raw product likelihood.
\end{itemize}
